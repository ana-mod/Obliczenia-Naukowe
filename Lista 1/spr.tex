\documentclass[11pt]{article}

\usepackage{amsmath}
\usepackage{polski}
\usepackage[utf8]{inputenc} 
\usepackage{listings}
\lstdefinelanguage{Julia}%
  {morekeywords={abstract,break,case,catch,const,continue,do,else,elseif,%
      end,export,false,for,function,immutable,import,importall,if,in,%
      macro,module,otherwise,quote,return,switch,true,try,type,typealias,%
      using,while},%
   sensitive=true,%
   alsoother={\$},%
   morecomment=[l]\#,%
   morecomment=[n]{\#=}{=\#},%
   morestring=[s]{"}{"},%
   morestring=[m]{'}{'},%
}[keywords,comments,strings]%
\usepackage{scrextend}
\usepackage{algorithm}
\usepackage{color}
\definecolor{dkgreen}{rgb}{0,0.6,0}
\definecolor{gray}{rgb}{0.5,0.5,0.5}
\definecolor{mauve}{rgb}{0.58,0,0.82}
\usepackage[noend]{algpseudocode}
\makeatletter
\def\BState{\State\hskip-\ALG@thistlm}
\makeatother
\usepackage[a4paper, total={6in, 8in}]{geometry}
\addtokomafont{labelinglabel}{\sffamily}
\usepackage[parfill]{parskip}
\renewcommand\thesection{Zadanie~\arabic{section}}
\renewcommand\thesubsection{\arabic{subsection})}
\lstdefinestyle{JuliaStyle}{language=Julia,
breaklines=true,
frame=tb,
  aboveskip=3mm,
  belowskip=3mm,
  showstringspaces=false,
  columns=flexible,
  basicstyle={\small\ttfamily},
  numbers=none,
  numberstyle=\tiny\color{gray},
  keywordstyle=\color{blue},
  commentstyle=\color{dkgreen},
  stringstyle=\color{mauve},
  breaklines=true,
  breakatwhitespace=true,
  tabsize=3}
\lstdefinestyle{BashStyle}{basicstyle=\ttfamily,
frame=single}
\lstset{
    inputencoding=utf8,
    extendedchars=true,
            literate={ą}{{\k{a}}}1
             {Ą}{{\k{A}}}1
             {ę}{{\k{e}}}1
             {Ę}{{\k{E}}}1
             {ó}{{\'o}}1
             {Ó}{{\'O}}1
             {ś}{{\'s}}1
             {Ś}{{\'S}}1
             {ł}{{\l{}}}1
             {Ł}{{\L{}}}1
             {ż}{{\.z}}1
             {Ż}{{\.Z}}1
             {ź}{{\'z}}1
             {Ź}{{\'Z}}1
             {ć}{{\'c}}1
             {Ć}{{\'C}}1
             {ń}{{\'n}}1
             {Ń}{{\'N}}1,
}
\usepackage{amsthm}
\usepackage{hyperref}
\theoremstyle{remark}
\newtheorem*{remark}{Uwaga}
\begin{document}
%ż
%\begin{lstlisting}
%Microsoft Windows [Version 6.1.7601]
%Copyright (c) 2009 Microsoft Corporation. All rights %reserved.żżółŁ
%c:\User\Garbage Collector>
%\end{lstlisting}
%\begin{align}
%f(x) &= x^2\\
%f'(x) &= 2x\\
%F(x) &= \int f(x)dx\\
%F(x) &= \frac{1}{3}x^3
%\end{align}
\begin{titlepage}
	\centering
	
	{\scshape\LARGE Politechnika Wrocławska \par}
	\vspace{1cm}
	{\scshape\Large Anna Modrzejewska\par}
		{\scshape\Large 236642\par}

	\vspace{1.5cm}
	{\huge\bfseries Obliczenia naukowe\par}
	\vspace{0.5cm}
	{\Large\itshape Lista nr 1\par}
	\vfill

% Bottom of the page
	{\large \today\par}
\end{titlepage}

\newpage
\newgeometry{
 total={170mm,257mm},
 left=20mm,
 top=20mm,
 }
\section{}
\subsection{Opis problemu}
Należy iteracyjnie wyznaczyć wartość epsilonu maszynowego $macheps$, liczbę $eta$ oraz liczbę $MAX$ w arytmetyce Float16, Float32 oraz Float64.
\subsection{Rozwiązanie}
Do wyznaczenia wartości $macheps$ można posłużyć się pętlą:
\begin{lstlisting}[style=JuliaStyle]
x = fl(1.0)
while fl(1.0+x/2)>1.0
	x=fl(x/2)
end
\end{lstlisting}
Do wyznaczenia liczby $eta$:
\begin{lstlisting}[style=JuliaStyle]
x = fl(1.0)
while fl(x/2)>0.0
	x=fl(x/2)
end
\end{lstlisting}
Do wyznaczenia liczby $MAX$ (funkcja $isinf(x)$ zwraca true w przypadku, gdy argument jest nieskończony, false w przeciwnym):
\begin{lstlisting}[style=JuliaStyle]
x = fl(1.0)
while !isinf(fl(x*2))
	x=fl(x*2)
end
x=x*(2-eps(fl))
\end{lstlisting}
\subsection{Otrzymane wyniki}
\begin{center}
\begin{tabular}{ |c|c|c|c| } 
 \hline
  & macheps & eta & max \\ 
  \hline
 Float16 & 0.000977 &  $6.0e^{-8}$ & $6.55e^4$\\ 
 Float32 & $1.1920929e^{-7}$ & $1.0e^{-45}$ & $3.4028235e^{38}$\\
 Float64 & $2.220446049250313e^{-16}$ & $5.0e^{-324}$ & $1.7976931348623157e^{308}$\\
 \hline
\end{tabular}
\end{center}
\subsection{Analiza wyników}
Otrzymane przez program wyniki są takie same, co wartości zwracane przez funkcje $eps()$, $nextfloat(fl(0.0))$, $realmax()$ oraz FLT\_EPSILON, DBL\_EPSILON, FLT\_MAX, DBL\_MAX z pliku nagłówkowego float.h języka C.

Zgodnie z definicją podaną na wykładzie, precyzją arytmetyki nazywa się największy błąd względny dla przedstawienia liczby w arytmetyce zmiennopozycyjnej (zaokrąglenia). Zatem wartość epsilonu maszynowego (która została wyżej wyznaczona jako najmniejsza taka liczba, która spełnia warunek $x+1.0>1.0$) równa się precyzji arytmetyki. 
\section{}
\subsection{Opis problemu}
Należy sprawdzić słuszność wzoru Kahana, według którego epsilon maszynowy można otrzymać obliczając wartość wyrażenia: $3*(\frac{4}{3}-1)-1$. 
\subsection{Otrzymane wyniki}
\begin{center}
\begin{tabular}{ |c|c|c| } 
 \hline
  & ze wzoru Kahana & z funkcji eps() \\ 
  \hline
 Float16 & -0.000977 &  0.000977 \\ 
 Float32 & 1.1920929e-7 & 1.1920929e-7\\
 Float64 & -2.220446049250313e-16 & 2.220446049250313e-16\\
 \hline
\end{tabular}
\end{center}
\subsection{Analiza wyników i wnioski}
Można zauważyć, że wyniki z wzoru Kahana są równe z epsilonem maszynowym co do modułu, zatem teza Kahana jest słuszna. Jednak otrzymane wartości różnią się od $0$, czego powodem może być to, że wartość ułamka $\frac{4}{3}$ w reprezentacji liczb zmiennoprzecinkowych nie jest dokładną wartością, tylko liczbą zbliżoną do $\frac{4}{3}$.
\section{}
\subsection{Opis problemu}
Należy sprawdzić, że w arytmetyce Float64 liczby z zakresu $[1,2]$ są rozmieszczone równomiernie z krokiem $\delta=2^{-52}$ oraz sprawdzić, jak rozmieszczone są w przedziałach $[\frac{1}{2}, 1]$ i $[2,4]$
\subsection{Rozwiązanie}
Program  przy pomocy funkcji $bits$ sprawdza zapis bitowy  liczby 1.0, następnej po 1.0 (przy użyciu funkcji nextfloat) oraz liczby $1+2^{-52}$, a także zapis bitowy liczby $2-2^{-52}$, następnej po niej i liczby 2.0.
\subsection{Otrzymane wyniki i analiza}
\begin{lstlisting}[style=BashStyle]
julia> bits(1.0)
0011111111110000000000000000000000000000000000000000000000000000
julia> bits(nextfloat(1.0)
0011111111110000000000000000000000000000000000000000000000000001
julia> bits(1.0+2.0^(-52))
0011111111110000000000000000000000000000000000000000000000000001
\end{lstlisting}
Można wywnioskować, że kolejna po 1.0 liczba jest oddalona od niej o $\delta=2^{-52}$. Podobnie z liczbą 2.0:
\begin{lstlisting}[style=BashStyle]
julia> bits(2.0)
0100000000000000000000000000000000000000000000000000000000000000
julia> bits(2.0-2.0^(-52))
0011111111111111111111111111111111111111111111111111111111111111
julia> bits(nextfloat(2.0-2.0^(-52))
0100000000000000000000000000000000000000000000000000000000000000
\end{lstlisting}
Widać, że po poprzednia liczba od 2 jest oddalona od niej o dokładnie deltę.
Dla przedziału $[\frac{1}{2}, 1]$ krokiem okazała się $\delta = 2^{-53}$, natomiast dla $[2, 4]$ $\delta = 2^{-51}$:
\begin{lstlisting}[style=BashStyle]
julia> bits(0.5)
0011111111100000000000000000000000000000000000000000000000000000
julia> bits(0.5+2.0^(-53))
0011111111100000000000000000000000000000000000000000000000000001
julia> bits(2.0)
0100000000000000000000000000000000000000000000000000000000000000
julia> bits(2.0+2.0^(-51))
0100000000000000000000000000000000000000000000000000000000000001
\end{lstlisting}
Można zauważyć, że dla przedziału $[2^m, 2^{m+1}]$ im większe m, tym większa delta, czyli mniejsze zagęszczenie liczb. Zmiana delty następuje w momencie zmiany cechy liczby (bity od 2. do 12.). Dla liczb z przedziału $[\frac{1}{2}, 1]$ cechą jest 01111111110, dla $[1, 2]$ jest to 01111111111, dla $[2, 4]$: 10000000000. Wzrost cechy o 1 powoduje dwukrotny wzrost delty.
\section{}
\subsection{Opis problemu}
Należy znaleźć w arytmetyce Float64 takie najmniejsze $x \in (1,2)$, że $x*\frac{1}{x} \neq 1$
\subsection{Rozwiązanie}
Program liczy wartość wyrażenia $x*\frac{1}{x}$ w pętli, biorąc za x kolejne wartości (za pomocą funkcji nextfloat()) do momentu, aż będzie się różniło od 1.
\subsection{Wynik}
Znaleziony x ma wartość $x=1.000000057228997$. Wartość podanego wyrażenia to 0.9999999999999999, a jego reprezentacja bitowa to 0011111111101111111111111111111111111111111111111111111111111111
\section{}
\subsection{Opis problemu}
Należy zaimplementować 4 algorytmy obliczające iloczyn skalarny dwóch wektorów w arytmetyce Float32 oraz Float64. Algorytmy różnią się kolejnością sumowania składników.
\subsection{Rozwiązanie}
Algorytm A liczy sumę składników ``wprzód", czyli dodaje od początku do końca. Algorytm B liczy sumę od końca do początku. Algorytm C tworzy nowy wektor $z=[x_1y_1, x_2y_2, x_3y_3, x_4y_4, x_5y_5$, sortuje rosnąco, a następnie liczby dodatnie sumuje od największej do najmniejszej, a ujemne od najmniejszej do największej. Algorytm D liczy sumę odwrotnie do algorytmu C.
\subsection{Wyniki}
\begin{center}
\begin{tabular}{ |c|c|c|c|c| } 
 \hline
  & Algorytm A & Algorytm B & Algorytm C & Algorytm D \\ 
  \hline
 Float32 & -0.4999443 & -0.4543457 & -0.5 &  -0.5\\
 błąd & 0.49994429944939167 & 0.4543457031149 & 0.4999999999899 & 0.4999999999899\\
 \hline
 Float64 & $1.02518813683*10^{-10}$ & $-1.5643308870494*10^{-10}$ & 0.0 & 0.0\\
 błąd & $1.1258452438296672*10^{-10}$ &  $1.4636737800494365*10^{-10}$ & $1.00657107*10^{-11}$ & $1.00657107*10^{-11}$\\
 \hline
\end{tabular}

\end{center}
Rzeczywista wartość iloczynu skalarnego: $-1.00657107000000*10^{-11}$
\subsection{Analiza wyników}
Najbliższe rzeczywistego wyniku są rezultaty algorytmu C i D liczące w arytmetyce Float64. Na podstawie różności wyników można wywnioskować, że kolejność sumowania ma znaczenie. Najlepiej sumować ze sobą w pierwszej kolejności najbliższe sobie rzędem wielkości, a na końcu najdalsze.
\section{}
\subsection{Opis problemu}
Należy policzyć w arytmetyce podwójnej precyzji wartości funkcji $f(x)=\sqrt{x^2+1}-1$ oraz $g(x)=\frac{x^2}{\sqrt{x^2+1}+1}$ dla $x=8^{-1}, 8^{-2} ...$. 
\subsection{Otrzymane wyniki}
Po wywołaniu funkcji $f(x)$ oraz $g(x)$ z powyższymi argumentami wyniki będą następujące:

\begin{lstlisting}[style=BashStyle]
julia> f(8.0^-1)
0.0077822185373186414

julia> g(8.0^-1)
0.0077822185373187065
\end{lstlisting}
\begin{lstlisting}[style=BashStyle]
julia> f(8.0^-2)
0.00012206286282867573

julia> g(8.0^-2)
0.00012206286282875901
\end{lstlisting}
\begin{lstlisting}[style=BashStyle]
julia> f(8.0^-3)
1.9073468138230965e-6

julia> g(8.0^-3)
1.907346813826566e-6
\end{lstlisting}
\begin{lstlisting}[style=BashStyle]
julia> f(8.0^-8)
1.7763568394002505e-15

julia> g(8.0^-8)
1.7763568394002489e-15
\end{lstlisting}
\begin{lstlisting}[style=BashStyle]
julia> f(8.0^-9)
0.0

julia> g(8.0^-9)
2.7755575615628914e-17
\end{lstlisting}
\begin{lstlisting}[style=BashStyle]
julia> f(8.0^-10)
0.0

julia> g(8.0^-10)
4.336808689942018e-19
\end{lstlisting}
\subsection{Analiza wyników i wnioski} Można zauważyć, że mimo $f=g$, wyniki dla tych samych argumentów nieznacznie różnią się, szczególnie dla $x\leq8^{-9}$, dla którego funkcja $f(x)$ zwraca 0.0. Wynika to prawdopodobnie z tego, że w przypadku f(x), odejmując 1 od liczby zbliżonej do 1 następuje redukcja znaczących cyfr.

\section{}
\subsection{Opis problemu}
Należy policzyć w arytmetyce Float64 przybliżoną wartość pochodnej funkcji $f(x)=sinx+cos3x$ w punkcie $x_0=1$ oraz błąd bezwzględny.
\subsection{Rozwiązanie}
Program składa się z funkcji $df$ wyliczającej pochodną z ogólnego wzoru $f'(x_0)=\frac{f(x_0+h)-f(x_0)}{h}$, funkcji $f(x)=sin(x)+cos(3x)$ oraz funkcji $df2$ liczącej rzeczywistą pochodną funkcji $f(x)$: $f'(x)=cos(x)-3sin(3x)$ w celu porównania wyników funkcji $df$ i $df2$.
\subsection{Otrzymane wyniki}
Funkcja $df$ została przetestowana dla $h=2^{-n}$, gdzie $n \in \{0, 1, 2, ..., 54\}$:
\begin{center}
\begin{tabular}{ |c|c|c| } 
 \hline
 h & wynik df & błąd \\ 
  \hline
 $2^0$ & 2.0179892252685967 &  1.9010469435800585\\ 
 $2^{-1}$ & 1.8704413979316472 & 1.753499116243109 \\
 $2^{-2}$ & 1.1077870952342974 &  0.9908448135457593 \\
 ... & ... & ... \\
 $2^{-20}$ & 0.11694612901192158 & 3.8473233834324105e-6 \\
 $2^{-21}$ & 0.1169442052487284 & 1.9235601902423127e-6 \\
 $2^{-22}$ & 0.11694324295967817 & 9.612711400208696e-7 \\
 ... & ... & ... \\
 $2^{-47}$ & 0.109375 & 0.007567281688538152 \\
 $2^{-48}$ & 0.09375 & 0.023192281688538152 \\
 $2^{-49}$ & 0.125 & 0.008057718311461848 \\
 $2^{-50}$ & 0.0 & 0.11694228168853815 \\
 $2^{-51}$ & 0.0 & 0.11694228168853815 \\
 $2^{-52}$ & -0.5 & 0.6169422816885382 \\
 $2^{-53}$ & 0.0 & 0.11694228168853815 \\
 $2^{-54}$ & 0.0 & 0.11694228168853815 \\
 \hline
\end{tabular}
\end{center}
\subsection{Analiza wyników i wnioski}
Można zauważyć, że dla $n\leq-50$ funkcja zaczyna zwracać tę samą wartość 0.0, ponieważ odejmujemy wartość $f(x_0)$ od liczby do niej zbliżonej (podobnie, jak w zadaniu wyżej - następuje redukcja znaczących cyfr), dlatego coraz mniejsze $h$ będzie generowało taki sam błąd. Wynika to z niewystarczającej precyzji arytmetyki.
\end{document}